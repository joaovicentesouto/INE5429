\documentclass{article}
\usepackage[utf8]{inputenc}
\usepackage[margin=1.2in]{geometry}

\title{%
  Atividade 2 \\
  Trabalho individual sobre Números Primos}
\author{João Vicente Souto \\ 16105151}
\date{14 abril 2019}

\usepackage{natbib}
\usepackage{graphicx}
\usepackage{multirow}

% Code
\usepackage{listings}
\usepackage{color}

\definecolor{dkgreen}{rgb}{0,0.6,0}
\definecolor{gray}{rgb}{0.5,0.5,0.5}
\definecolor{mauve}{rgb}{0.58,0,0.82}

\lstset{frame=tb,
  language=Python,
  aboveskip=3mm,
  belowskip=3mm,
  showstringspaces=false,
  columns=flexible,
  basicstyle={\small\ttfamily},
  numbers=none,
  numberstyle=\tiny\color{gray},
  keywordstyle=\color{blue},
  commentstyle=\color{dkgreen},
  stringstyle=\color{mauve},
  breaklines=true,
  breakatwhitespace=true,
  tabsize=3
}
% End Code

\begin{document}

\maketitle

\section{Introdução}

Este trabalho apresenta a implementação de dois algoritmos para geração de números aleatórios e, utilizando o mais eficiente deles, é apresentado mais dois algoritmos para testar a primalidade dos números aleatórios gerados.

Para facilitar o trabalho de implementação, principalmente na manipulação de números grande, foi utilizado a linguagem de programação \textit{Python}.
Desta forma, o código gerado se torna mais intuitivo apesar da perda de desempenho em comparação com linguagens de mais baixo nível, \textit{C/C++} por exemplo.

Primeiramente, a Seção \ref{sec:random} abordará a geração de números primos e a justificativa na escolha dos algoritmos, assim como, o detalhamento dos mesmo e a comparação entre as implementações.
Em seguida, a Seção \ref{sec:prime} apresentará os teste de primalidade, apresentando os algoritmos, suas implementações e suas comparações.
A Seção \ref{sec:conclusao} conclui o trabalho e, por fim, a Seção \ref{sec:codigo} apresenta os códigos fontes completos de cada script utilizado.

\section{Geradores de números aleatórios}
\label{sec:random}

Como mencionado em aula, a geração de números aleatórios é de extrema importância em diversas áreas da computação, como segurança, simulação etc.
Porém, gerar um número aleatório real é inviável em muitos casos, desta forma, recorre-se a algoritmos que geram sequências de números pseudo-aleatórios.
Ou seja, gerar números que apresentem certas características que impossibilite um atacante descobrir a sequência na qual ele pertence.

Neste contexto, foram escolhidos dois algoritmos amplamente utilizados e conhecidos para geração de números aleatórios descritos a seguir.
Como base, foram utilizados o livro \cite{bib:livro} disponível no Moodle e artigos disponíveis na Wikipedia.

\subsection{\textit{Linear congruential generator (LCG)}}
\label{lcg-parcial}

O gerador congruencial linear foi escolhido porque é um dos mais antigos e mais conhecidos algoritmos que produz sequências de números pseudo-aleatórios calculados com uma função linear \cite{bib:lcg1}.

O gerador é definido pela relação de recorrência:

{
\centering{$X_{n+1} = (a * X_n + c) mod m$ onde:}
    \begin{itemize}
	    \item $X$ é a sequência de valores pseudo-aleatórios
	    \item $m$ é o módulo, sendo $0 < m$
	    \item $a$ é o multiplicador, sendo $0 < a < m$
	    \item $c$ é o incremento, sendo $0 <= c < m$
	    \item $X_0$ é os valor inicial, sendo $0 <= X_0 < m$ 
    \end{itemize}
}
    
Outro fator que levaram a sua escolha foi a possibilidade de utilizar um valor arbitrário no módulo $m$, desta forma, um número poderá ser calculado dentro de um intervalo de $0$ à $2^{4096}$, por exemplo.

O código a seguir demonstra a criação de uma classe que inicializa os parâmetros mencionados acima. Onde a função \textit{next()} implementa a relação de recorrência.

\begin{lstlisting}
# Algoritmo LCG:
# LinearCongruentialGenerators.py (Parcial)

class LinearCongruentialGenerator:
	def __init__(self, m, a, c, x0):
		"""
		Configure the initial state of the RNG.

		Parameters:
		m  -- The modulus    : m > 0
		a  -- The multiplier : 0 < a < m 
		c  -- The increment  : 0 <= c < m
		x0 -- The seed       : 0 <= x0 < m
		"""
		assert(m > 0)
		assert(0 < a and a < m)
		assert(0 <= c and c < m)
		assert(0 <= x0 and x0 < m)

		self.m  = m
		self.a  = a
		self.c  = c
		self.xi = x0

	def next(self):
		"""
		Generates the next random number.

		Returns:
		The next random number
		"""
		# Generates the next random number with the
		# formula: X(i+1) = (a * X(i) + c) mod m
		self.xi = (self.a * self.xi + self.c) % self.m
		return self.xi
\end{lstlisting}

% \pagebreak
A saída, Figura \ref{fig:lcg-partial}, demonstra a execução do LCG utilizando os seguintes valores\cite{bib:lcg2}:
\begin{itemize}
	    \item $a = 1103515245$
	    \item $c = 12345$
	    \item $m = 2^{[40, 56, 80, 128, 168, 224, 256, 512, 1024, 2048, 4096]}$
	    \item $X_0 = tempo atual do sistema em milisegundos$
\end{itemize}

\begin{figure}[h!]
	\centering
	\includegraphics[scale=0.35]{figs/lcg-py.png}
	\caption{Saída do terminal rodando o script LinearCongruentialGenerators.py}
	\label{fig:lcg-partial}
\end{figure}

\subsection{\textit{Blum Blum Shub} (BBS)}

O \textit{Blum Blum Shub} é um algoritmo para geração de números pseudo-aleatórios proposto por Lenore Blum, Manuel Blum e Michael Shub em 1986 \cite{bib:bbs1}.
A motivo para sua escolha se assemelha ao algoritmo anterior, sendo ele muito conhecido e utilizado. Além disso, sua implementação é simples e permite lidar facilmente com números grandes\cite{bib:livro}.

O algoritmo BBS é:

{
\centering{$X_{n+1} = (X_n)^2 mod n$ onde:}
    \begin{itemize}
	    \item $n = p * q$, sendo:
	    \begin{itemize}
	        \item $p$ é primo e congruente a $3 (mod 4)$
	        \item $q$ é primo e congruente a $3 (mod 4)$
	        \item $p mod 4 = q mod 4 = 3$
	    \end{itemize}
	    \item $X$ é a sequência de valores pseudo-aleatórios
    \end{itemize}
}

Ao utilizar este algoritmo, é possível extrair uma sequência aleatória de bits, onde a probabilidade do próximo bit da sequência ser $0$ ou $1$ é igual a $1/2$, tornando o algoritmo mais robusto.

O código a seguir demonstra a criação de uma classe que inicializa os parâmetros mencionados acima. Onde a função \textit{next()} implementa o algoritmo BBS.

\pagebreak
\begin{lstlisting}
# Algoritmo BBS:
# BlumBlumShub.py (Parcial)

class BlumBlumShub:
	def __init__(self, n, seed, bits_amount):
		"""
		Configure the initial state of the RNG.

		Parameters:
		n            -- n = p * q where (p mod 4) = (q mod 4) = 3
		seed         -- Initial value
		bits_amount  -- Size of the bitstream for generate a number
		"""
		self.n = n
		self.bits_amount = bits_amount
		self.xi = pow(seed, 2, n)

	def next(self):
		"""
		Generates the next random number.
		"""
		# Bitstream
		bitstream = ""

		# Generates a stream with size equal bits amount
		for bits in range(0, self.bits_amount):

			# Calculates: X(i+1) = X(i)^2 % n
			self.xi = pow(self.xi, 2, self.n)

			# Gets the least bit significant
			bit_i = self.xi % 2

			# Concatenates the new bit on bitstream
			bitstream = bitstream + str(bit_i)
		
		# Converts bitstream into integer with bits amount of length
		return int(bitstream, 2)
\end{lstlisting}

A saída, Figura \ref{fig:blum-partial}, demonstra a execução do BBS utilizando os seguintes valores mencionados no livro fonte\cite{bib:bbs1}:
\begin{itemize}
	    \item $n = 192649$
	    \item $seed = 101355$
	    \item $bits\_amount = [40, 56, 80, 128, 168, 224, 256, 512, 1024, 2048, 4096]$
\end{itemize}

\pagebreak

\begin{figure}[h!]
	\centering
	\includegraphics[scale=0.35]{figs/blum-py.png}
	\caption{Saída do terminal rodando o script BlumBlumShub.py}
	\label{fig:blum-partial}
\end{figure}

\subsection{Comparação}

Ambos os algoritmos são similares em seus algoritmos, principalmente pela sua simplicidade, não necessitando de muito processamento de corrente das suas complexidade,  \texit{O(1)}.
Porém, o algoritmo LCG possui menor segurança por possibilitar que um atacante descubra quais são seus parâmetros através da análise de frequência dos números gerados.
Sabendo alguns parâmetros, é possível calcula-los a partir da resolução de um sistema linear.
Como alternativa é possível, a partir de um número de números gerados, alterar aleatoriamente os parâmetros do algoritmo, aumentando sua segurança.

Por outro lado, o algoritmo BBS gera uma sequência de bits aleatória entre si, ou seja, não é possível dizer qual será o valor do próximo bit gerado.
Ou seja, a probabilidade descobrir o próximo valor é de $1/2$, sendo que existem apenas dois valores possíveis.

Como pode ser analisado na Tabela \ref{tab:random}, o algoritmo LCG apresentou um desempenho significativo em comparação com o algoritmo BBS.
Isto se deve a implementação em \textit{Python} do BBS, onde para gerar um número, primeiramente é construído uma \textit{string} do comprimento desejado.
Desta forma, o maior tempo de desempenho é dado pela geração de n números aleatórios para conseguir calcular os n bits solicitados.

\pagebreak

\begin{table}[]
\centering
\begin{tabular}{|L|L|L|} \hline
Quantidade de bits    & Algoritmo & Tempo (\mu~segundos) \\ \hline \hline
\multirow{2}{*}{40}   & LCG       & 0.00444              \\ \cline{2-3}
					  & BBS       & 0.10157              \\ \hline
\multirow{2}{*}{56}   & LCG       & 0.00715              \\ \cline{2-3}
					  & BBS       & 0.12803              \\ \hline
\multirow{2}{*}{80}   & LCG       & 0.00558              \\ \cline{2-3}
					  & BBS       & 0.13351              \\ \hline
\multirow{2}{*}{128}  & LCG       & 0.00578              \\ \cline{2-3}
					  & BBS       & 0.23699              \\ \hline
\multirow{2}{*}{168}  & LCG       & 0.00539              \\ \cline{2-3}
					  & BBS       & 0.29516              \\ \hline
\multirow{2}{*}{224}  & LCG       & 0.00531              \\ \cline{2-3}
					  & BBS       & 0.37766              \\ \hline
\multirow{2}{*}{256}  & LCG       & 0.00546              \\ \cline{2-3}
					  & BBS       & 0.39947              \\ \hline
\multirow{2}{*}{512}  & LCG       & 0.00374              \\ \cline{2-3}
					  & BBS       & 0.80109              \\ \hline
\multirow{2}{*}{1024} & LCG       & 0.00682              \\ \cline{2-3}
					  & BBS       & 1.65081              \\ \hline
\multirow{2}{*}{2048} & LCG       & 0.00625              \\ \cline{2-3}
					  & BBS       & 3.91674              \\ \hline
\multirow{2}{*}{4096} & LCG       & 0.01071              \\ \cline{2-3}
					  & BBS       & 7.14870              \\ \hline
\end{tabular}

\caption{Comparação de tempos dos algoritmos de geração de números aleatórios}
\label{tab:random}
\end{table}

\section{Teste de Primalidade}
\label{sec:prime}

Isso é uma introdução

\subsection{Miller Rabin}

Isso é uma introdução

\begin{lstlisting}
# Algoritmo Miller Rabin:
# MillerRabin.py

def MillerRabin(n, k, rng):
	# Ignore some cases
	if n < 4 or n % 2 == 0:
		return "conclusive"

	# Write n as (2^r * d) + 1 with d odd (by factoring out powers of 2 from n - 1)
	d = n-1
	r = 0
	while d % 2 == 0:
		d = d >> 1
		r += 1

	# Repeat k times:
	for _ in range(k):
		
		# Pick a random integer a in the range [2, n - 2]
		a = (generate(rng) % n - 3) + 2

		# a^d % m
		x = pow(a, d, n)

		# With that a is inconclusive
		if x == 1 or x == n-1:
			continue

		conclusive = True
		for _ in range(r - 1):

			# x(i+1) = x(i)^2 % n
			x = pow(x, 2, n)

			# inconclusive
			if x == n - 1:
				conclusive = False
				continue
		
		# If this a is inconclusive then repeat the process
		if conclusive:
			return "conclusive"
	
	return "probably prime"
\end{lstlisting}

\begin{figure}[h!]
	\centering
	\includegraphics[scale=0.35]{figs/miller-rabin.png}
	\caption{Saída do terminal rodando o script MillerRabin.py}
	\label{fig:universe}
\end{figure}

\subsection{Fermat}

Isso é uma introdução

\begin{lstlisting}
# Algoritmo Fermat:
# Fermat.py

def Fermat(n, k, rng):
	# Ignore some cases
	if n < 2 or n % 2 == 0:
		return "conclusive"

	# Repeat k times
	for _ in range(k):

		# Generates a into [2, n - 2]
		a = (generate(rng) % n - 3) + 2

		# Calculate a^(n - 1) % n and if is different of 1 then n is not prime
		if pow(a, n - 1, n) != 1:
			return "conclusive"

	return "probably prime"
\end{lstlisting}

\begin{figure}[h!]
	\centering
	\includegraphics[scale=0.35]{figs/fermat.png}
	\caption{Saída do terminal rodando o script Fermat.py}
	\label{fig:universe}
\end{figure}

\subsection{Comparação}

Isso é uma introdução

\begin{table}[]
\centering
\begin{tabular}{|L|L|L|L|} \hline
Quantidade de bits    & Algoritmo    & Tempo (\mu~segundos) & Tentativas \\ \hline \hline
\multirow{2}{*}{40}   & Miller Rabin & 5.89514              & 25         \\ \cline{2-4}
					  & Fermat       & 6.08468              & 45         \\ \hline
\multirow{2}{*}{56}   & Miller Rabin & 2.13313              & 18         \\ \cline{2-4}
					  & Fermat       & 1.94001              & 2          \\ \hline
\multirow{2}{*}{80}   & Miller Rabin & 5.20706              & 25         \\ \cline{2-4}
					  & Fermat       & 7.13492              & 47         \\ \hline
\multirow{2}{*}{128}  & Miller Rabin & 10.43010             & 60         \\ \cline{2-4}
					  & Fermat       & 13.48829             & 36         \\ \hline
\multirow{2}{*}{168}  & Miller Rabin & 19.68312             & 33         \\ \cline{2-4}
					  & Fermat       & 18.60309             & 91         \\ \hline
\multirow{2}{*}{224}  & Miller Rabin & 30.15494             & 48         \\ \cline{2-4}
					  & Fermat       & 29.83402             & 74         \\ \hline
\multirow{2}{*}{256}  & Miller Rabin & 38.53512             & 105        \\ \cline{2-4}
					  & Fermat       & 32.87292             & 79         \\ \hline
\multirow{2}{*}{512}  & Miller Rabin & 600.76118            & 1066       \\ \cline{2-4}
					  & Fermat       & 673.37084            & 1219       \\ \hline
\multirow{2}{*}{1024} & Miller Rabin & 2336.81107           & 634        \\ \cline{2-4}
					  & Fermat       & 1455.97386           & 324        \\ \hline
\multirow{2}{*}{2048} & Miller Rabin & 3351.92204           & 5          \\ \cline{2-4}
					  & Fermat       & 44112.44226          & 1877       \\ \hline
\multirow{2}{*}{4096} & Miller Rabin & 95790.78412          & 584        \\ \cline{2-4}
					  & Fermat       & 455806.92720         & 3871       \\ \hline
\end{tabular}

\caption{Comparação de tempos dos algoritmos de teste de primalidade}
\label{tab:prime}

\end{table}

\section{Conclusão}
\label{sec:conclusao}

Ab e c

\pagebreak

\section{Códigos Fonte}
\label{sec:codigo}

\subsection{Código e teste do Linear Congruential Generator}
\label{lcg-full}

\begin{lstlisting}
# LinearCongruentialGenerators.py
import math, time

# ================ Implementation ================

import math, time

class LinearCongruentialGenerator:
	def __init__(self, m, a, c, x0):
		"""
		Configure the initial state of the RNG.

		Parameters:
		m  -- The modulus    : m > 0
		a  -- The multiplier : 0 < a < m 
		c  -- The increment  : 0 <= c < m
		x0 -- The seed       : 0 <= x0 < m
		"""
		assert(m > 0)
		assert(0 < a and a < m)
		assert(0 <= c and c < m)
		assert(0 <= x0 and x0 < m)

		self.m  = m
		self.a  = a
		self.c  = c
		self.xi = x0

	def next(self):
		"""
		Generates the next random number.

		Returns:
		The next random number
		"""
		# Generates the next random number with the
		# formula: X(i+1) = (a * X(i) + c) mod m
		self.xi = (self.a * self.xi + self.c) % self.m
		return self.xi

# ================ Auxiliar functions ================

def correct_bits_length(bits_expected, number):
	"""
	Checks if the number has sufficient bits

	Keyword arguments:
	bits_expected -- Number of bits expected. 
	number        -- Decimal number

	Returns:
	True if equal False otherwise
	"""
	length_expected = int(math.log10(2 ** bits_expected))
	actual_length = int(math.log10(number))
	return length_expected == actual_length

def generate(bits, a, c, x0):
	"""
	Builds the RNG and measure the time of generate a random number.

	Parameters:
    bits -- Number of bits expected
	a    -- The multiplier
	c    -- The increment
	x0   -- The seed

    Returns:
    A random number
	"""
	# Calculates the module m of the algorithm
	m = 2 ** bits

	# Init RNG with its arguments
	lcg = LinearCongruentialGenerator(m, a, c, x0)

	# Start measuring time
	start = time.time()

	# Genereates a random number
	number = lcg.next()
	generations = 1

	# While the number does not have sufficient bits
	while not correct_bits_length(bits, number):
		# Genereates a new random number
		number = lcg.next()
		generations += 1

	# Stop measuring time
	end = time.time()

	mean = (end - start) / generations * 1000.0

	# Print the results
	print("Generates a number with %d bits taking %.5f ms:" % (bits, mean))
	print("%d\n" % number)

# ================ Tests ================

if __name__ == '__main__':
	"""
	Main function to execute all random number generations
	"""

	# Default arguments
	a  = 1103515245
	c  = 12345
	x0 = int(time.time())
	bits_array = [40, 56, 80, 128, 168, 224, 256, 512, 1024, 2048, 4096]

	print("Linear Congruential Generator Parameters:")
	print("a : %d | c: %d | x0: %d | bits: %s\n" % (a, c, x0, str(bits_array)))

	# Execute random number generation
	for bits in bits_array:
		generate(bits, a, c, x0)
\end{lstlisting}

\pagebreak

\subsection{Código e teste do Blum Blum Shub}
\label{bbs-full}

\begin{lstlisting}
# BlumBlumShub.py
import math, time

# ================ Implementation ================

class BlumBlumShub:
	def __init__(self, n, seed, bits_amount):
		"""
		Configure the initial state of the RNG.

		Parameters:
		n            -- n = p * q where (p mod 4) = (q mod 4) = 3
		seed         -- Initial value
		bits_amount  -- Size of the bitstream for generate a number
		"""
		self.n = n
		self.bits_amount = bits_amount
		self.xi = pow(seed, 2, n)

	def next(self):
		"""
		Generates the next random number.
		"""
		# Bitstream
		bitstream = ""

		# Generates a stream with size equal bits amount
		for bits in range(0, self.bits_amount):

			# Calculates: X(i+1) = X(i)^2 % n
			self.xi = pow(self.xi, 2, self.n)

			# Gets the least bit significant
			bit_i = self.xi % 2

			# Concatenates the new bit on bitstream
			bitstream = bitstream + str(bit_i)
		
		# Converts bitstream into integer with bits amount of length
		return int(bitstream, 2)

# ================ Auxiliar functions ================

def correct_bits_length(bits_expected, number):
	"""
	Checks if the number has sufficient bits

	Keyword arguments:
	bits_expected -- Number of bits expected. 
	number        -- Decimal number

	Returns:
	True if equal False otherwise
	"""
	length_expected = int(math.log10(2 ** bits_expected))
	actual_length = int(math.log10(number))
	return length_expected == actual_length

def generate(n, seed, bits):
	"""
	Builds the RNG and measure the time of generate a random number.

	Parameters:
	n            -- n = p * q where (p mod 4) = (q mod 4) = 3
	seed         -- Initial value
	bits_amount  -- Size of the bitstream for generate a number
	"""
	# Init RNG with its arguments
	bbs = BlumBlumShub(n, seed, bits)

	# Start measuring time
	start = time.time()

	# Genereates a random number
	number = bbs.next()
	generations = 1

	# While the number does not have sufficient bits
	while not correct_bits_length(bits, number):
		# Genereates a new random number
		number = bbs.next()
		generations += 1
	
	# Stop measuring time
	end = time.time()

	mean = (end - start) / generations * 1000.0

	# Print the results
	print("Generates a number with %d bits taking %.5f ms:" % (bits, mean))
	print("%d\n" % number)

# ================ Tests ================

if __name__ == '__main__':
	"""
	Main function to execute all random number generations
    """

	# Default arguments
	n    = 192649
	seed = 101355
	bits_array = [40, 56, 80, 128, 168, 224, 256, 512, 1024, 2048, 4096]

	print("BlumBlumShub Parameters: n=%d, seed=%d, bits=%s\n" % (n, seed, str(bits_array)))

	# Execute random number generation
	for bits in bits_array:
		generate(n, seed, bits)
\end{lstlisting}

\pagebreak

\subsection{Código e teste do Miller Rabin}
\label{miller-rabin-full}

\begin{lstlisting}
# MillerRabin.py

import math, time
from LinearCongruentialGenerators import LinearCongruentialGenerators

# ================ Implementation ================

def MillerRabin(n, k, rng):
	# Ignore some cases
	if n < 4 or n % 2 == 0:
		return "conclusive"

	# Write n as (2^r * d) + 1 with d odd (by factoring out powers of 2 from n - 1)
	d = n-1
	r = 0
	while d % 2 == 0:
		d = d >> 1
		r += 1

	# Repeat k times:
	for _ in range(k):
		
		# Pick a random integer a in the range [2, n - 2]
		a = (generate(rng) % n - 3) + 2

		# a^d % m
		x = pow(a, d, n)

		# With that a, is inconclusive
		if x == 1 or x == n-1:
			continue

		conclusive = True
		for _ in range(r - 1):

			# x(i+1) = x(i)^2 % n
			x = pow(x, 2, n)

			# inconclusive
			if x == n - 1:
				conclusive = False
				continue
		
		# If this a is inconclusive then repeat the process
		if conclusive:
			return "conclusive"
	
	return "probably prime"

# ================ Auxiliar functions ================

def correct_bits_length(bits_expected, number):
	"""
	Checks if the number has sufficient bits

	Keyword arguments:
	bits_expected -- Number of bits expected. 
	number        -- Decimal number

	Returns:
	True if equal False otherwise
	"""
	length_expected = int(math.log10(2 ** bits_expected))
	actual_length = int(math.log10(number))
	return length_expected == actual_length

def generate(rng):
	"""
	Builds the RNG and measure the time of generate a random number.

	Parameters:
	rng -- RNG

    Returns:
    A random number
	"""
	# Genereates a random number
	number = rng.next()

	# While the number does not have sufficient bits
	while not correct_bits_length(bits, number):
		# Genereates a new random number
		number = rng.next()
	
	return number

# ================ Tests ================

if __name__ == '__main__':

	iterations = 100
	bits_array = [40, 56, 80, 128, 168, 224, 256, 512, 1024, 2048, 4096]

	print("MillerRabin with Linear Congruential Generators (%d iterations)\n" % iterations)

	# For any amount of bits
	for bits in bits_array:
		number = 0
		attempts = 0
		is_prime = False

		# Use LCG as RNG
		rng = LinearCongruentialGenerators(2 ** bits, 25214903917, 11, int(time.time()))

		# Start measuring time
		start = time.time()

		# While the number is not prime
		while not is_prime:

			# Attempts counter
			attempts+= 1
			is_prime = True

			# Generates a random number
			number = generate(rng)

			# Checks if is probably a prime number
			is_prime = MillerRabin(number, iterations, rng) == "probably prime"

		# Stop measuring time
		end = time.time()

		final_time = (end - start) * 1000.0

		print("Bits %d: took %d attempts and took %.5f ms to find the follow prime number:" % (bits, attempts, final_time))
		print("%d\n" % number)
\end{lstlisting}

\pagebreak

\subsection{Código e teste do Fermat}
\label{fermat-full}

\begin{lstlisting}
# Fermat.py

import math, time
from LinearCongruentialGenerators import LinearCongruentialGenerators

# ================ Implementation ================

def Fermat(n, k, rng):
	# Ignore some cases
	if n < 2 or n % 2 == 0:
		return "conclusive"

	# Repeat k times
	for _ in range(k):

		# Generates a into [2, n-2]
		a = (generate(rng) % n - 3) + 2

		# Calculate a^(n-1) % n and if is different of 1 then n is not prime
		if pow(a, n - 1, n) != 1:
			return "conclusive"

	return "probably prime"

# ================ Auxiliar functions ================

def correct_bits_length(bits_expected, number):
	"""
	Checks if the number has sufficient bits

	Keyword arguments:
	bits_expected -- Number of bits expected. 
	number        -- Decimal number

	Returns:
	True if equal False otherwise
	"""
	length_expected = int(math.log10(2 ** bits_expected))
	actual_length = int(math.log10(number))
	return length_expected == actual_length

def generate(rng):
	"""
	Builds the RNG and measure the time of generate a random number.

	Parameters:
    rng -- RNG

    Returns:
    A random number
	"""
	# Genereates a random number
	number = rng.next()

	# While the number does not have sufficient bits
	while not correct_bits_length(bits, number):
		# Genereates a new random number
		number = rng.next()

	return number

# ================ Tests ================

if __name__ == '__main__':

	iterations = 100
	bits_array = [40, 56, 80, 128, 168, 224, 256, 512, 1024, 2048, 4096]

	print("Fermat with Linear Congruential Generators (%d iterations)\n" % iterations)

	for bits in bits_array:
		number = 0
		attempts = 0
		is_prime = False
		rng = LinearCongruentialGenerators(2 ** bits, 25214903917, 11, int(time.time()))

		start = time.time()

		while not is_prime:

			attempts+= 1
			is_prime = True
			number = generate(rng)

			is_prime = Fermat(number, iterations, rng) == "probably prime"

		end = time.time()

		final_time = (end - start) * 1000.0

		print("Bits %d: took %d attempts and took %.5f ms to find the follow prime number:" % (bits, attempts, final_time))
		print("%d\n" % number)
\end{lstlisting}

\pagebreak

\begin{thebibliography}{5}

\bibitem{bib:lcg1}
Definição e algoritmo: Linear congruential generator
\\\texttt{https://en.wikipedia.org/wiki/Linear\_congruential\_generator}

\bibitem{bib:lcg2}
Parâmetros e algoritmo: Linear congruential generator
\\\texttt{https://rstudio-pubs-static.s3.amazonaws.com/300131\_e504d6ca49714187990520fb22e86607.html}

\bibitem{bib:bbs1} 
Definição e algoritmo: Blum Blum Shub
\\\texttt{https://en.wikipedia.org/wiki/Blum\_Blum\_Shub}

\bibitem{bib:miller-rabin}
Algoritmo: Miller Rabin
\\\texttt{https://rosettacode.org/wiki/Miller\%E2\%80\%93Rabin\_primality\_test\#Python}

\bibitem{bib:fermat}
Definição e algoritmo: Fermat
\\\texttt{https://en.wikipedia.org/wiki/Fermat\_primality\_test}

\bibitem{bib:livro}
Definição e algoritmo: LCG, BBS, Miller Rabin e Fermat: Livro fonte no Moodle
\\\texttt{Stallings, William - Cryptography and Network Security: Principles and Practice, Sixth Edition}

\end{thebibliography}

\end{document}
